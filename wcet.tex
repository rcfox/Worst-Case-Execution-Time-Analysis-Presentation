\documentclass{beamer}

\mode<presentation>
{
  \usetheme{Darmstadt}
}

\usepackage[english]{babel}
\usepackage[latin1]{inputenc}
\usepackage{url}

\title{Worst-Case Execution Time Analysis}

\author{Ryan Fox}

\institute
{
  Department of Electrical and Computer Engineering\\
  University of Waterloo
}

\date{\today}

\AtBeginSubsection[]
{
  \begin{frame}<beamer>{Outline}
    \tableofcontents[currentsection,currentsubsection]
  \end{frame}
}

\begin{document}

\begin{frame}
  \titlepage
\end{frame}

\begin{frame}{Outline}
  \tableofcontents[pausesections]
\end{frame}

\section{Introduction}
\subsection{Definitions}

\begin{frame}{WCET}
  \begin{itemize}
    \item $WCET_A$ -- Actual Worst Case Execution Time
    \item $WCET_D$ -- Derived Worst Case Execution Time
  \end{itemize}
\end{frame}

\begin{frame}{Safety}
  Is $WCET_D \geq WCET_A$ ?
  \begin{itemize}
    \item WCET Bound
      \begin{itemize}
        \item $WCET_A \in (0, WCET_D]$
        \item Safe
      \end{itemize}      
      \pause
    \item WCET Estimate
      \begin{itemize}
        \item Maximum value of all sampled WCETs
        \item Does not consider all possible cases
        \item Not safe          
      \end{itemize}      
  \end{itemize}
\end{frame}

\begin{frame}{Precision}
  \begin{itemize}
    \item Are the bounds or estimates close to the actual values? \\
      \pause
    \item Less Precision
      \begin{itemize}
      \item[$\Rightarrow$] More idle time
      \item[$\Rightarrow$] Harder to schedule
      \end{itemize}
  \end{itemize}
\end{frame}

\subsection{Assumptions}

\begin{frame}{Assumptions}
  \pause
  \begin{itemize}
    \item Hard Real-Time System
    \item Execution does not continue indefinitely
    \item Loops and recursion are explicitly bounded
    \item No dynamic jumps
      \begin{itemize}
        \item Function pointers
        \item \texttt{switch} statements
      \end{itemize}
    \item No interruptions
  \end{itemize}
\end{frame}

\section{Static Analysis}
\subsection{Front-End}
\begin{frame}{Front-End}
  \begin{columns}
    \begin{column}{0.6\textwidth}
      \begin{itemize}
        \item<1-> Input:
          \begin{itemize}
            \item Source code; or
            \item Machine code; and maybe
            \item User annotations
          \end{itemize}
        \item<2-> Output:
          \begin{itemize}
            \item Call Graph
            \item Control-Flow Graph
            \item Other Information
              \begin{itemize}
              \item Ranges for input data
              \item Iteration/Recursion bounds
              \end{itemize}
          \end{itemize}
      \end{itemize}
    \end{column}
    \begin{column}{0.4\textwidth}
      \begin{center}
        \begin{itemize}
          \item<1->[] \includegraphics[scale=0.3]{code.png}
          \item<2->[] \includegraphics[scale=0.2]{controlflow.png}
        \end{itemize}
      \end{center}      
    \end{column}
  \end{columns}
\end{frame}

\subsection{Control-Flow Analysis}
\begin{frame}{Control-Flow Analysis}
  \begin{columns}
    \begin{column}{0.63\textwidth}
      \begin{itemize}
        \item Finds all possible paths of execution
        \item Attempts to eliminate infeasible paths to improve precision
          \begin{itemize}
            \item \texttt{if (a \&\& !a)}
          \end{itemize}
        \item Determines bounds of iteration and recursion
        \item<2-> Challenges:
          \begin{itemize}
            \item Dynamic jumps
            \item Pipe-lining
            \item Out-of-order execution
          \end{itemize}
      \end{itemize}
    \end{column}
    \begin{column}{0.37\textwidth}
      \includegraphics[scale=0.35]{controlflowanalysis.png}
    \end{column}
  \end{columns}
\end{frame}

\subsection{Processor Behaviour Analysis}
\begin{frame}{Processor Behaviour Analysis}
  \begin{itemize}
    \item Uses a \underline{model} of the processor
    \item Determines the timing for each task
      \begin{itemize}
        \item Instruction
        \item Sequence of instructions
      \end{itemize}
    \item \alert{Context is important}
      \begin{itemize}
        \item Memory, Cache
        \item Pipelines, Branch Prediction
        \item<3-> Time Anomalies
          \begin{itemize}
            \item<3-> ie: Cache misses due to branch mis-predictions
          \end{itemize}
      \end{itemize}
      \item<2-> The obvious worst-case may not be the actual worst-case
  \end{itemize}
\end{frame}

\subsection{Bound Calculation}
\begin{frame}{Bound Calculation}  
  \begin{itemize}
    \item Combines individual tasks' timings to get overall $WCET_D$
      \pause
    \item Three main approaches:
      \begin{itemize}
        \item Path-Based Calculation
        \item Structure-Based Calculation
        \item Implicit Path Enumeration Technique (IPET)
      \end{itemize}
  \end{itemize}
\end{frame}

\begin{frame}{Example Input}
  \begin{columns}
    \begin{column}{0.25\textwidth}
      \includegraphics[scale=0.3]{input.png}
    \end{column}
    \begin{column}{0.75\textwidth}
      \begin{itemize}
        \item Nodes represent basic elements of the task
        \item Edges represent possible execution paths
        \item Values beside nodes indicate individual WCETs
        \item For the following examples:
          \begin{itemize}
            \item Less than 100 loop iterations
          \end{itemize}
        \item {\footnotesize{Examples adapted from Wilhelm et al. 2007.}}
      \end{itemize}
    \end{column}
  \end{columns}
\end{frame}

\begin{frame}{Structure-Based Bound Calculation}
  \begin{itemize}
    \item Bottom-up traversal of the syntax tree
    \item A set of rules is given for combining child nodes with parents
      \begin{itemize}
        \item Combine all possibilities of a conditional
        \item Combine sequential nodes
        \item Etc.
      \end{itemize}
    \item $T_{New} =  T_{Parent} + max(T_{Child0}, T_{Child1},\dots)$
    \item Simple and computationally cheap
    \item Inter-node dependencies or optimized code may cause problems
  \end{itemize}
\end{frame}

\begin{frame}{Structure-Based Bound Calculation Example}
  \begin{columns}
    \begin{column}{0.25\textwidth}
      \includegraphics[scale=0.3]{structureoutput1.png}
    \end{column}
    \begin{column}{0.25\textwidth}
      \includegraphics[scale=0.3]{structureoutput2.png}
    \end{column}
    \begin{column}{0.25\textwidth}
      \includegraphics[scale=0.3]{structureoutput3.png}
    \end{column}
    \begin{column}{0.25\textwidth}
      \includegraphics[scale=0.3]{structureoutput4.png}
    \end{column}
  \end{columns}
\end{frame}

\begin{frame}{Path-Based Bound Calculation}  
    \begin{columns}
    \begin{column}{0.25\textwidth}
      \includegraphics[scale=0.3]{pathoutput.png}
    \end{column}
    \begin{column}{0.75\textwidth}
      \begin{itemize}
        \item Each possible path is traversed, and $WCET_{PATHi}$ is calculated
        \item $WCET_D = max(WCET_{PATH0},WCET_{PATH1},\dots)$
          \pause
        \item Longest execution path is explicitly shown.
          \begin{itemize}
            \item Best target for optimizations
          \end{itemize}
        \item Analysis time rises exponentially with the number of branches
      \end{itemize}
    \end{column}
  \end{columns}
\end{frame}

\begin{frame}[IPET]{Implicit Path Enumeration Technique}
  \begin{itemize}
    \item Each node has a time value, $t$, and a counter variable, $x$
    \item Program structure is represented as constraining linear equations
    \item $WCET_D$ can be found using either:
      \begin{itemize}
      \item Integer Linear Programming
        \begin{itemize}
          \item More popular method
          \item Efficient solvers exist
        \end{itemize}
      \item Constrain Satisfaction
        \begin{itemize}
          \item Allows for constraints that are more complex
          \item Generally slower than integer linear programming
        \end{itemize}
      \item In either case, the problem is NP-hard
      \end{itemize}
  \end{itemize}
\end{frame}

\begin{frame}{IPET Example}
  \begin{columns}
    \begin{column}{0.25\textwidth}
      \includegraphics[scale=0.3]{ipetoutput.png}
    \end{column}
    \begin{column}{0.75\textwidth}
      \begin{eqnarray*}
        x_{start} &=& 1 \\
        x_{end} &=& 1 \\
        x_A &<& 100 \\
        x_A &=& x_{start,A} + x_{H,A} = x_{A,end} + x_{A,B} \\
        x_B &=& x_{A,B} = x_{B,C} + x_{B,D} \\
        x_C &=& x_{B,C} = x_{C,E} \\
        &\dots& \\
        x_H &=& x_{F,H} + x_{G,H} = x_{H,A} \\
        x_{end} &=& x_{A,end} \\
        WCET_D &=& maximize(\sum x_it_i)
      \end{eqnarray*}
    \end{column}
  \end{columns}
\end{frame}

\section{Measurement}
\subsection{Measurement-Based Analysis}
\begin{frame}{General}
  \begin{itemize}
    \item Results gathered experimentally by executing the task
    \item Generally not feasible to measure all possible cases of execution
    \item Provides an \underline{estimate} of the WCET
    \item Not safe for hard real-time purposes
      \pause
    \item Compatible with the static analysis framework
      \begin{itemize}
        \item Measurements may substitute for the Processor Behaviour Analysis
        \item May also overlap with Bound Calculations
      \end{itemize}
  \end{itemize}
\end{frame}

\begin{frame}{Why Bother?}
  \begin{itemize}
    \item Safety might not be a concern
    \item Static analysis can be slow
    \item Average-case execution time emerges as a side-effect
    \item \alert{Exercises your code}
  \end{itemize}
\end{frame}

\subsection{Methods}
\begin{frame}{Measurement on Hardware}
  \begin{itemize}
    \item Intrusive
      \begin{itemize}
        \item Collect timestamps during execution
        \item CPU cycle counter
      \end{itemize}
    \item Non-Intrusive
      \begin{itemize}
        \item Logic Analyzer
      \end{itemize}
  \end{itemize}
\end{frame}
  
\begin{frame}{Measurement on Simulator}
  \begin{itemize}
    \item Execute the task on a virtual processor
    \item Advantages:
      \begin{itemize}
        \item Physical hardware not necessary
        \item Many simulations can be run in parallel
        \item Intrusive measurements not necessary
      \end{itemize}
    \item Disadvantages:
      \begin{itemize}
        \item May not be 100\% accurate to processor behaviour
      \end{itemize}
  \end{itemize}
\end{frame}

\section{Summary}
\subsection{Static vs. Measurement-Based Analysis}

\begin{frame}{Static vs. Measurement-Based Analysis}
  \begin{columns}
    \begin{column}{0.5\textwidth}
      Static Analysis
      \begin{itemize}
        \item[$\checkmark$] Calculates bounds for WCET
        \item[$\checkmark$] Safe
        \item[$\checkmark$] No special hardware needed
        \item[$X$] Processor-specific
        \item[$X$] Usually over-estimates WCET
      \end{itemize}
    \end{column}
    \begin{column}{0.5\textwidth}
      \pause
      Measurements
      \begin{itemize}
        \item[$\checkmark$] Simple to apply to a variety of hardware
        \item[$X$] Calculates estimates of the WCET
        \item[$X$] Generally unsafe
        \item[]
        \item[]
      \end{itemize}
    \end{column}
  \end{columns}
\end{frame}

\subsection{References}
\begin{frame}{References}
  \begin{itemize}
    \item The Worst-Case Execution Time Problem -- Overview of Methods and Survey of Tools
      \begin{itemize}
        \item Wilhelm et. al, 2007
        \item \url{http://moss.csc.ncsu.edu/\~mueller/ftp/pub/mueller/papers/1257.pdf}
      \end{itemize}
    \item A Modular Tool Architecture for Worst-Case Execution Time Analysis
      \begin{itemize}
        \item Ermedahl, 2003
        \item \url{http://www.mrtc.mdh.se/publications/0570.pdf}
      \end{itemize}
  \end{itemize}
\end{frame}

\end{document}
